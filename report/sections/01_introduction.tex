\chapter{Introduction} \label{sec:Intro}

\section{Overview and Motivation} \label{sec:Intro_Overview}

The video game industry is hugely popular and rapidly growing. In 2020, the industry generated over 179 billion US dollars in global revenue, an increase of 20\% over the previous year \cite{GamingIndustrySize}. The Steam platform, a video game distribution service, is the largest of its kind for PC users \cite{SteamLargestDistributor} with over 120 million active monthly users \cite{SteamMonthlyUsers} and a selection of over 60 thousand games \cite{SteamGameCount}.

This dissertation will make use of a large collection of data taken from the Steam platform via its public API. This dataset primarily consists of extensive information regarding the game reviews that users have written; however, it also contains social information about the reviewers, such as who they are friends with and the gaming communities that they are involved in.

The primary motivations behind this project are twofold. The first involves determining whether the written text of Steam reviews can be used by machine learning models to predict other features of the review, and, if so, to what extent are these predictions accurate or reliable. The review features being predicted might include the review's polarity/sentiment; its helpfulness, as determined by other users; or the amount of time the reviewer had spent playing the game prior to writing their review. The second motivation involves the development of a method to identify particularly representative users among the entire population using only the written text of the reviews. A representative user is considered to be someone whose ratings or opinions of games tend to align with the population as a whole, thus acting as a reliable and valuable source of information as to how a new game, or any product, for that matter, might be received by other users.

The dataset will be comprehensively examined prior to designing and implementing the above approaches. This examination will make use of plots and visualisations in order to provide a deeper understanding of the nature of the reviews, the users and the Steam platform itself. Some of the dataset characteristics that will be examined are the number of reviews that users write, the languages in which reviews are written, the lengths of the reviews, the proportion of reviews that are positive or negative, the number of reviews that are considered helpful by other users, and the lengths of time that reviewers spend playing the games they review. This examination will provide crucial insight into the underlying structure of both Steam and the dataset which will help inform and guide the design and implementation of the aforementioned approaches.

The first research motivation will utilise state-of-the-art language representation techniques, such as BERT, to investigate the predictive validity of models trained using only the review text. Detailed analysis will be performed to determine the effect that specific text features have on the models' results. These text features might include the word count, the language the review was written in or how balanced the distribution of the output feature being predicted is. The results of the experiments, both positive and negative, will be discussed and a number of potential improvements will be suggested.

The second research motivation will involve building upon the results of the first, ie whether or not the review text is informative and can be used to make reliable predictions about other features. The employed approach will involve the development of a machine learning classifier that will use review texts to predict the overall rating of the game being reviewed. The resulting model, assuming it is found to be sufficiently accurate, will be evaluated for each individual reviewer in order to determine those users whose combinations of review text and game ratings the model is most capable of predicting accurately. Different methods of selecting the most representative users will be examined and compared. The methodology behind this approach, which seems to be somewhat novel, will be explained and justified; the assumptions made during its design, as well as the numerous drawbacks that it presents, will also be discussed. A myriad of potential ways to improve upon or refine the approach will be suggested. Finally, a selection of real world, commercial applications of this research, mainly related to recommender systems, will be proposed.

This research appears to be well suited to platforms such as Steam due to the gaming industry's rapid increase in popularity and tremendous economic value. As a result of the abundance of new games being released, a method of quickly and reliably determining whether or not a new game is worth recommending to users would be of considerable value to platforms like Steam as a means of increasing total sales as well as user retention and satisfaction. This determination could be done using methods based on those that will be outlined in this report.

\section{Dissertation Structure} \label{sec:Intro_Structure}

This dissertation will consist of eight chapters, including this one. In chapter \ref{sec:LR}, an overview of the existing literature will be provided. In chapter \ref{sec:BG}, the structure and functionality of both the Steam platform and the dataset will be described in greater detail. Chapter \ref{sec:Dataset} will consist of an in depth examination of the dataset, accompanied by numerous visualisations. In chapter \ref{sec:TBG}, technical descriptions and explanations will be provided for each of the machine learning methods, techniques and evaluation metrics utilised throughout the dissertation. Chapter \ref{sec:DI} will provide a thorough description of the approaches used to achieve the research objectives, while chapter \ref{sec:Res} will present and discuss the results of these approaches. In chapter \ref{sec:Conc}, the dissertation will be concluded, with the overall results of the research being summarised and assessed.
